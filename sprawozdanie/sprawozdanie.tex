\documentclass[a4paper,12pt]{article}
\usepackage[utf8]{inputenc}
\usepackage[polish]{babel}
\usepackage{polski}
\usepackage{amsmath}
\usepackage{a4wide}
\usepackage{url}
\pagestyle{plain}

\makeatletter
  \renewcommand{\@seccntformat}[1]{\csname the#1\endcsname.\quad}
\makeatother

\title{\textbf{Wyszukiwanie informacji} \\ pracownia, zadanie nr 1}
\author{
Wiktor Janas \thanks{\texttt{wixorpeek@gmail.com}}
\and
Mateusz Dereniowski \thanks{\texttt{Mateusz.Dereniowski@gmail.com}}}

\begin{document}

\maketitle
\thispagestyle{empty}

\section{O projekcie}
Niniejsza praca dokumentuje projekt polegający na stworzeniu wyszukiwarki dla
polskiej Wikipedii. Projekt obejmuje stworzenie indeksera, czyli programu
przetwarzającego tekst Wikipedii do postaci ułatwiającej wyszukiwanie, oraz
samej wyszukiwarki, czyli programu odpowiadającego na zapytania.

Wyszukiwarka obsługuje zapytania boolowskie oraz frazowe, jak również umożliwia
wyszukiwanie z lematyzacją. W realizacji projektu wykorzystano udostępniony
tekst Wikipedii\footnote{\url{http://www.ii.uni.wroc.pl/~prych/IR/wikipedia_dla_wyszukiwarek.txt.gz}},
morfologik\footnote{\url{http://www.ii.uni.wroc.pl/~prych/text\_mining/bazy_do_tm.txt}}
oraz listę \textit{stopwords}\footnote{\url{http://www.ii.uni.wroc.pl/~prych/text_mining/stop_words.txt}}.

Projekt został zrealizowany przy użyciu języków C++ oraz Python. Wymagany jest
system operacyjny Linux wraz z biblioteką \texttt{talloc} (zazwyczaj wystarczy
zainstalować pakiet \texttt{libtalloc-dev}). Aby skompilować programy należy
wydać polecenie \texttt{make} w katalogu z kodem źródłowym. Następnie można
przystąpić do tworzenia indeksu: w tem celu należy wydać polecenie
\texttt{./create-index.sh}, którego argumentami powinny być kolejno:
ścieżka do tekstu Wikipedii, ścieżka do morfologika oraz ścieżka do pliku
\textit{stopwords} (uwaga: pliki te powinny mieć zakończenia linii w formacie
UNIX; z dostarczonych materiałów plik \textit{stopwords} wymaga konwersji,
można jej dokonać używając programu \texttt{fromdos}). Wraz z projektem
dostarczony jest utworzony indeks. Wreszcie można uruchomić wyszukiwarkę
wydając polecenie \texttt{./search}. Wyszukiwarka domyślnie pracuje w trybie
wsdadowym: wczytuje zapytania ze standardowego wejścia i wypisuje wyniki na
standardowe wyjście; można ją zatem przetestować używając polecenia
\texttt{./search < queries.txt > answers.txt}. Wyszukiwarka posiada tryb
interaktywny, aktywowany opcją \texttt{-v}. W trybie tym wyszukiwarka wypisuje
zachęte do wprowadzenia zapytania oraz wiele informacji o przetwarzaniu.
Ponieważ w trybie interaktywnym wynikowa lista tytułów artykułów często jest
zbędna, wyszukiwarka posiada również opcję \texttt{-r}, która wyłącza
wypisywanie listy wyników (informacja o ilości trafień wciąż jest wypisywana).

\section{Ciekawe rozwiązania zastosowane w projekcie}
Wśród wielu standardowych rozwiązań zastosowanych w projekcie (parser zapytań,
algorytmy obliczania sumy i iloczynu list postingowych) znalały się też takie,
które wydają się nam szczególnie ciekawe. Zostały one opisane poniżej.

\subsection{Zamiana słów na liczby}
Rozwiązanie problemu zamiany słów na liczby jest podstawową kwestią wymagającą
rozwiązania w wyszukiwarce: operacje na słowach są nieefektywne, słowami nie
można na przykład indeksować tablicy (,,tablica'' umożliwiająca taką operację
jest w istocie słownikiem i czas dostępu do niej często jest niezadawalający).
Z jednej strony wejściem dla indeksera jest wielki zbiór słów, z drugiej strony
wyszukiwarka powinna możliwie błyskawicznie parsować zapytanie.

Standardowym podejściem do problemu byłoby posortowanie wszystkich znanych słów
leksykograficznie, a następnie zastosowanie wyszukiwania binarnego w celu
odnalezienia indeksu poszukiwanego słowa w tablicy. Inną (równoważną)
możliwością byłoby umieszczenie słów w zbalansowanym drzewie poszukiwań
binarnych. Oba te podejścia charakteryzuje jednak logarytmiczny czas dostępu do
słowa (względem ilości słów). Ponadto wymagają one dość chaotycznych dostępów
do pamięci, co redukuje efektywność cache'a procesora. Uzyskanie stałego czasu
dostępu byłoby możliwe z wykorzystaniem struktury trie, jest ona jednak bardzo
pamięciożerna.

Rozwiązanie zastosowane w tej pracy opiera się na tabeli hashującej.
Podczas budowania indeksu wszystkie słowa są hashowane i dzielone na kubełki.
W kubełku ustalana jest arbitralna kolejność słów. Następnie wszystkie kubełki
są konkatenowane: w ten sposób każde słowo otrzymuje swój unikalny
identyfikator z przedziału od zera do $\text{ilości słów} - 1$. Wyszukiwanie
numeru słowa polega na obliczeniu jego hasha oraz odnalezienia numeru szukanego
słowa w kubełku. Aby obliczyć identyfikator wystarczy numer ten dodać do
łącznej ilości słów we wszystkich poprzedzających kubełkach (wielkość tą
oblicza się prostym preprocessingiem i zapamiętuje).

\subsection{Lematyzacja}
Celem lematyzacji jest uniezależnienie wyników zapytania od form gramatycznych
użytych przez użytkownika i autora tekstu. W związku z tym, w odpowiedzi na term
wpisany przez użytkownika chcemy zwrócić dokument, w którym występuje
jakiekolwiek słowo, które ma wspólną formę bazową z termem z zapytania. Prostą
realizacją tego pomysłu byłoby: zindeksuj każdy term z Wikipedii jako wszystkie
jego formy bazowe, zaś przy wykonywaniu zapytania potraktuj term z zapytania
jako sumę (or) wszystkich jego form bazowych. Niestety takie podejście narzuca
na wyszukiwarkę konieczność zamienienia każdego termu na odpowiednią sumę,
obliczania tych sum oraz, co najgorsze, odczytywania z dysku wielu list
postingowych.

W niniejszej pracy zastosowano inne rozwiązanie. Przypuśćmy, że dla każdego
termu znajdujemy wszystkie jego formy bazowe, zaś dla każdej z nich znajdujemy
z kolei wszystkie formy pochodne. Otrzymane zbiory sumujemy i term początkowy
indeksujemy jako każdy z termów z wynikowej sumy. Na przykład dla termu
,,kota'' formy bazowe to ,,kot'' oraz ,,kota''. Pochodnymi słowa ,,kot'' są
,,kocie'', ,,kotowi'', ,,kotów'' itd., zaś pochodnymi słowa ,,kota'' są
na przykład ,,kotą'' oraz ,,koto''. Zatem słowo ,,kota'' zostałoby zindeksowane
jako każde z powyższych słów.

Opisane powyżej podejście zwalnia wyszukiwarkę z obowiązku sprawdzania form
bazowych w morfologiku oraz obliczania sum; niesie jednak za sobą koszt
niebagatelnego zwiększenia rozmiaru słownika: zwykły rzeczownik odmienia się
przecież przez siedem przypadków i dwie liczby, zatem zostałby zindeksowany
przynajmniej czternaście razy! Problem ten jednak da się rozwiązać korzystając
z następującej obserwacji: prawie wszystkie formy pochodne jednego rzeczownika
mają tylko jedną formę bazową. Zatem wszystkie one będą miały dokładnie te same
wpisy na listach lematyzowanych. Można zatem pamiętać tylko jedną taką listę
i zapisać w indeksie, że wszystkie formy tego rzeczownika współdzielą ją.
Rozwiązanie to nie usuwa całkowicie problemu duplikacji: w morfologiku może
wystąpić sytuacja, gdzie bazą dla \texttt{a} jest \texttt{A}, zaś bazą dla
\texttt{b} jest \texttt{A} oraz \texttt{B}. W takim razie wystąpienie termu
\texttt{a} powinno zostać zindeksowane jako wystąpienie \texttt{a} oraz
\texttt{b}, jednak wystąpienie \texttt{b} liczy się jedynie jako wystąpienie
\texttt{b}. Widać zatem, że nie każdy term zostanie zindeksowany dokładnie raz.
Nie stanowi to jednak poważnego problemu: łatwo zauważyć, że sytuacja taka może
wystąpić jedynie w przypadku słów, które mają przynajmniej dwie formy bazowe,
słów takich jest zaś niewiele. Dzięki powyższej optymalizacji rozmiar indeksu
nie rośnie istotnie, zaś wyszukiwarka nie musi martwić się lematyzacją.

\subsection{Wczytywanie list w tle}
Najwolniejszą operacją w systemie komputerowym jest przesuwanie głowicy
dysku. Żądania odczytu generowane przez wyszukiwarkę są niestety bardzo
rozproszone (listy pozycyjne odpowiadające termom w zapytaniu prawie
nigdy nie znajdują się obok siebie). Przeto przez lwią część czas
wyszukiwarka oczekuje na dane z dysku. Jest to okazja do optymalizacji:
wyszukiwarka może zlecić wykonanie odczytów innemu wątkowi; kiedy tylko
dostarczy on jakąś listę pozycyjną może ona przystąpić do jej przetwarzania,
zaś ów wątek może bez zwłoki zażądać od systemu odczytu kolejnej listy
postingowej. Ponieważ czas przetwarzania listy znacznie dominuje nad czasem
odczytu z dysku, powinno dać się go ,,ukryć'' tak, aby nie miał wpływu na
całkowity czas wykonywania zapytania.

W wyszukiwarce zastosowano opisaną powyżej optymalizację. Okazuje się jednak,
że jej wpływ nie jest znaczący. Częściowo jest to spowodowane tym, że wiele
zapytań składa się z dwóch lub trzech słów. Ponieważ główny wątek potrzebuje
przynajmniej dwóch list, aby wykonać jakieś obliczenia, w przypadku krótkich
zapytań może okazać się, że przetwarzanie można rozpocząć dopiero po odczytaniu
z dysku wszystkich zażądanych list.

Inną możliwą do wykonania optymalizacją jest zażądanie od systemu odczytu
\textit{wielu} list jednocześnie. Należy przypuszczać, że spowodowałoby
to znaczny spadek czasu odczytu z dysku, gdyż system mógłby wysłać wiele
żądań do dysku, ten zaś mógłby zoptymalizować trasę głowicy nad tależami
(mechanizmy takie istnieją i są wykorzystywane w zaawansowanych aplikacjach).
Tej optymalizacji jednak nie zaimplementowano.

\subsection{Optymalizator}
Przed przystąpieniem do wykonywania zapytania boolowskiego uruchamiany jest
na nim optymalizator. Operuje on na drzewie zapytania i potrafi dokonać
następujących przekształceń:
\begin{itemize}
 \item usuwanie termów o pustych listach postingowych (jeśli term taki
 pojawi się w klauzuli AND, cała klauzula jest od razu zamieniana na
 pustą listę postingową; w klauzuli OR term jest po prostu usuwany),
 \item usuwanie \textit{stopwords} (\textit{stopwords} są usuwane jedynie
 z klauzul AND oraz jedynie gdy ich ilość nie przekracza połowy ilości
 termów w zapytaniu),
 \item sklejanie takich samych termów (wykonywanie operacji AND lub
 OR na takich samych termach nie ma sensu, zduplikowane wystąpienia 
 termu są usuwane; porównywanie dokonywane jest przy użyciu identyfikatora
 listy postingowej; ze względu na mechanizm lematyzacji termy w zapytaniu
 \texttt{kotu|kotom} mają ten sam identyfikator i jeden z nich zostanie
 usunięty),
 \item obliczanie optymalnej kolejności łączenia list (używany jest
 algorytm Huffmana, przy czym dla operacji OR przyjmuje się, że
 rozmiar wyniku będzie sumą rozmiarów danych wejściowych, zaś dla
 klauzuli AND -- minimum).
\end{itemize}
Dla zapytań frazowych optymalizator znajduje jedynie optymalną kolejność
łączenia list: od najkrótszych do najdłuższych.
Pomimo wykonania powyższych optymalizacji, w zapytaniu wciąż mogą pojawić
się wielokrotnie te same listy postingowe, na przykład w zapytaniu
\texttt{(pies kot) | (słoń pies|kot)}. Znajdowany jest więc zbiór
unikalnych list postingowych tak, aby każdą listę odczytać z dysku
tylko raz.

\section{Budowanie indeksu}
Proces budowania indeksu podzielony jest na kilka etapów; każdy z nich jest
wykonywany przez osobny program, które przekazują sobie dane przy użyciu plików
pośrednich.

\paragraph{Tokenizacja Wikipedii} Dostarczone dane wejściowe są zamieniane
na format, w którym każdy artykuł jest zapisany w jednej linii składającej się
z kolejnych słów występujących w artykule oddzielonych spacjami. Na tym etapie
usuwane są wszystkie zbędne znaki, dokonywana jest zamianana liter na małe
oraz formowane są termy. Na tym etapie powstaje również lista tytułów
artykułów (plik \texttt{TITL}).

\paragraph{Tworzenie korpusu} Na podstawie ztokenizowanej wikipedii oraz
morfologika tworzony jest korpus. Jest to plik o formacie podobnym do
słownika. Stanowi on tabelę hashującą zawierającą termy Wikipedii oraz
morfologika. Dzięki temu plikowi w dalszej części procesu możliwa będzie
łatwa zamiana słów na liczby. Należy zauważyć, że w pliku tym umieszczane są
wszystkie słowa, z jakimi kiedykolwiek może spotkać się indekser.

\paragraph{Parsowanie morfologika} Przy użyciu korpusu tworzona jest binarna
wersja morfologika: zawiera ona takie same informacje, jak zwykły morfologik
jednak jest prostsza w obsłudze, ponieważ wszystkie termy zamienione zostają
na ich identyfikatory. Ponadto morfologik zostaje uzupełniony o słowa
znajdujące się w korpusie, a nie znajdujące się w oryginalnym morfologiku:
dla słow tych przyjmuje się, że są one swoją postacią bazową.

\paragraph{Wykrywanie aliasów} Na podstawie binarnego morfologika tworzona
jest baza aliasów: dla każdego termu określane jest, czy powinien on mieć
swoją własną listę lematyzowaną, czy też powinien korzystać z listy innego
termu (czyli być aliasem). Jeśli term powinien być aliasem, określane jest,
czyjej listy lematyzowanej powinien używać.

\paragraph{Budowanie grafu lematyzacji} Na podstawie binarnego morfologika
oraz bazy aliasów tworzony jest graf lematyzacji: dla każdego termu budowana
jest lista termów, których listy lematyzowane powinny zawierać jego
wystąpienia. Na listę tą trafiają jedynie te termy, które nie są aliasami.

\paragraph{Digitizacja} Ztokenizowana Wikipedia zamieniana jest na format
binarny, to znaczy kolejne tokeny zamieniane są na trójki (\texttt{id\_tokenu},
\texttt{nr\_dokumentu}, \texttt{nr\_słowa\_w\_dokumencie}), gdzie wartość
\texttt{id\_tokenu} wyznaczana jest przez korpus.

\paragraph{Lematyzacja} Na postawie zdigitizowanej Wikipedii oraz grafu
lematyzacji przeprowadzana jest lematyzacja: każdy term zdigitizowanej
Wikipedii jest zamieniany na te termy, których listy lematyzowane powinny
złapać jego wystąpienie.

\paragraph{Odwracanie} Zdigitizowana Wikipedia oraz zlematyzowana Wikipedia
są sortowane leksykograficznie, przez co powstają zaczątki indeksu odwróconego.

\paragraph{Tworzenie indeksów i słownika} Na podstawie korpusu oraz indeksów
odwróconych powstałych w poprzednim kroku powstaje słownik (plik
\texttt{DICT}), indeks pozycyjny (plik \texttt{IDXP}) oraz indeks zlematyzowany
(plik \texttt{IDXL}). Na tym etapie usuwane są słowa, które nie mają żadnych
postingów (mogły one znaleźć się w korpusie, gdyż występowały w morfologiku,
jednak żaden term Wikipedii nie ,,aktywował'' ich).

\paragraph{Oznaczanie stopwords} W gotowych plikach indeksu na każde
\textit{stopword} nanoszony jest specjalny znacznik, który jest rozpoznawany
przez wyszukiwarkę.

\end{document}

